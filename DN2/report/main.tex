\documentclass[11pt]{article}
% THE PREAMBLE
% {

% Packages used.

\usepackage[a4paper, margin=2cm]{geometry} % Allows setting margins.
\usepackage[utf8]{inputenc}   % UTF-8 encoding.
\usepackage[T1]{fontenc} % Font encodings.
\usepackage{titling} % For title formating.
\usepackage{tikz} % For grids.
\usepackage{xcolor} % For colours.
\usepackage{amsmath} % Most math commands.
\usepackage{amssymb} % For math symbols.
\usepackage{enumitem} % For enumerating items.
\usepackage{hyperref} % Linking documents and websites.
\usepackage{import} % Importing ipynb code and other files.
\usepackage{float} % Most often used to stop figures from moving.
\usepackage{tabularx} % More complex tables.
\usepackage[english]{babel}
\usepackage{longtable}
\usepackage{tcolorbox} % For the coloured boxes used in pseudocode.
\usepackage{multicol}
\usepackage{multirow} 
\usepackage[table]{colortbl}
\usepackage{titlesec}
\usepackage{minted} % For code highlights ect.
\usepackage{fontspec}

\setmonofont{DejaVu Sans Mono}[Scale=0.90] % Or Consolas, JetBrains Mono, Source Code Pro, etc.

\emergencystretch 3em
\setlength{\parindent}{0pt} % No indent.
\setlength{\parskip}{10pt} % White space between paragraphs.

% Environment for centering.
% Setting the centering so it doesn't have too much space at the start and end.
\BeforeBeginEnvironment{center}{\setlength{\parskip}{0pt}} 
\AfterEndEnvironment{center}{\setlength{\parskip}{\baselineskip}}

% Environment for mint.
\BeforeBeginEnvironment{minted}{
  \setlength{\parskip}{0pt}
  \setlength{\topsep}{0pt}
  \setlength{\partopsep}{0pt}
}

% Hyperlink settup.
\hypersetup{
    colorlinks=true,    
    urlcolor=black!60!blue,
    citecolor=black!60!blue,
    linkcolor=black!60!blue
}

% Adjusting titles, sections and subsections.
\titlespacing*{\section}{0pt}{2pt}{4pt}  % Adjusts spacing for \section
\titlespacing*{\subsection}{0pt}{3pt}{3pt}  % Adjusts spacing for \subsection
\setlength{\droptitle}{-5em} % Making the title a little higher.
\renewcommand{\maketitle}{%
     \begin{center}
    {\LARGE\bfseries Analyzing Algorithms for the Subset Sum Problem}\\[0.5em]  
    {\large Nina Mislej \hspace{0.9em} \today}\\[0.3em]  
  \end{center}
  \vspace{-1.5em} % Reduce space after title
}

\newcommand{\cpp}{C\raisebox{0.4ex}{\scalebox{0.8}{++}} }

\begin{document}
\maketitle

The entire code can be found on the \href{https://github.com/Edelwy/approximation-algorithms}{\textbf{GitHub repositroy}}. The below mentioned results analyzed and raw can also be found in the repository as well.

\section{Implementation of the Algorithms}

I chose to implement the algorithms in \cpp because it provides sufficient built-in functionality and flexibility, while remaining faster than many other languages. Firstly, here is the implementation of the algorithm used for \textbf{dynamic programming}. The \texttt{memo} table is used for memoization and is initialized to $0$ for every element.

\begin{minted}[fontsize=\fontsize{10.5pt}{11pt}, 
               bgcolor=gray!10, 
               breaklines=true]{cpp}

  bool CSolver::solveDYN( int n, int k, const std::vector<int>& numbers )
  {
      std::vector<std::vector<int>> memo( n, std::vector<int>( k, 0 ) );
      std::function<int( int, int )> solver = [&]( int i, int j ) -> int {
          if( i < 0 || j < 0 || i > n || j > k )
              return std::numeric_limits<int>::min();
  
          if( i == 0 || j == 0 )
              return 0;
  
          auto& mem = memo.at( i - 1 ).at( j - 1 );
          if( mem > 0 ) 
              return mem;
  
          const auto& num = numbers.at(i - 1);
          mem = std::max( solver( i - 1, j ), solver( i - 1, j - num ) + num );
          return mem;
      };
      auto solution = solver( n, k );
      std::cout << fmt::format( "Solution found was {}.\n", solution );
      return solution == k;
  }
\end{minted}

Next algorithm is the \textbf{exhaustive search}. We do not actually add lists together but keep all elements in one set, which is sorted by default in \cpp standard library implementation. An optimization would be using the \texttt{unordered\_set} and just checking the existance of $k$ in the set at the end. We would, however, lose the information on the maximum sum not exceding $k$ without sorting.

\begin{minted}[fontsize=\fontsize{10.5pt}{11pt}, 
    bgcolor=gray!10, 
    breaklines=true]{cpp}

  bool CSolver::solveEXH( int n, int k, const std::vector<int>& numbers )
  {
      std::set<int> sums = {0};
      for ( int i = 1; i < n; i++ ) {
          auto currSums = sums;
  
          for ( const auto element : currSums ) {
              auto newElement = element + numbers.at(i);
              if ( newElement <= k )
                  sums.insert(newElement);
          }
      }
      auto solution = *sums.rbegin();
      std::cout << fmt::format( "Maximal element in set is {}.\n", solution );
      return solution == k;
  } 
\end{minted}

\pagebreak
The implementation of the \textbf{greedy algorithm} is optimized by sorting the numbers array first. If we frame this as a maximization problem, we are searching for the maximum sum not exceding $k$. Then this is a $0.5$-approximation, meaning the approximation is at least half as good as the actual solution.

\begin{minted}[fontsize=\fontsize{10.5pt}{11pt}, 
    bgcolor=gray!10, 
    breaklines=true]{cpp}

  bool CSolver::solveGRDY( int n, int k, const std::vector<int>& numbers )
  {
      auto sortedNumbers = numbers;
      std::sort(sortedNumbers.begin(), sortedNumbers.end());
  
      int solution = 0;
      for ( int i = 0; i < n; i++ ) {
          const auto& element = sortedNumbers.at( i );
          if (k - solution >= element )
              solution += element;
      }
      auto approx = fmt::format( "Approximation for {} is {}.\n", k, solution );
      approx += fmt::format( "Difference is {}.\n", k - solution );
      std::cout << approx;
      return solution == k;
  }  
\end{minted}

Finally we have the \textbf{FPTAS algorithm} which is very similar to the exahustive search with addition of trimming the list, or the set in our case. The trimming is based on the $\epsilon$ parameter.

\begin{minted}[fontsize=\fontsize{10.5pt}{11pt}, 
    bgcolor=gray!10, 
    breaklines=true]{cpp}

  bool CSolver::solveFPTAS( int n, int k, const std::vector<int>& numbers )
  {
      auto sortedNumbers = numbers;
      std::sort(sortedNumbers.begin(), sortedNumbers.end());
      auto delta = mEpsilon / (2 * n);
  
      std::set<int> sums = {0};
      for ( int i = 1; i < n; i++ ) {
          auto tmpSums = sums;
  
          for ( const auto element : tmpSums ) 
              sums.insert( element + sortedNumbers.at(i) );
  
          auto last = *sums.begin();
          tmpSums = { last };
          for ( auto& element : sums ) {
              if( element <= k && element > last * ( 1 + delta ) ) {
                  tmpSums.insert( element );
                  last = element;
              }
          }
          sums = tmpSums;
      }
  
      auto solution = *sums.rbegin();
      auto approx = fmt::format( "Approximation for {} is {}. \n", k, solution);
      approx += fmt::format( "The epsilon value was {}.", mEpsilon );
      approx += fmt::format( "Difference is {}.\n", k - solution );
      std::cout << approx;
      return solution == k;
  }  
\end{minted}

All four algorithms are functions of the \texttt{Solver} class. Which algorithm is used is parsed from parameters and decided based on the enum value.

\begin{minted}[fontsize=\fontsize{10.5pt}{11pt}, 
    bgcolor=gray!10, 
    breaklines=true]{cpp}
  enum class EMode { DYN = 1, EXH = 2, GRDY = 3, FPTAS = 4 };
\end{minted}

\section{Test Case Generation for Hard Scenarios}



\end{document}